\section{Discussion}
\label{sec:discussion}
The expert survey results provide compelling evidence for the high perceptual realism of brightfield microscopy images generated by our unconditional diffusion model.
The near-chance level accuracy achieved by microscopy experts in distinguishing synthetic images from real ones strongly support the notion that diffusion models can effectively synthesize microscopy data that is visually indistinguishable from real-world acquisitions.
This finding directly addresses our first sub-research question and highlights the potential of diffusion models to generate data suitable for augmenting or even substituting real microscopy images in certain applications.

Object detection experiments further reveal the practical utility of diffusion-based synthetic data for single cell detection.
The comparable, and in some cases, slightly improved performance of models trained with synthetic data augmentation at mAP\@50 demonstrates that synthetic images can effectively capture essential cell features and spatial distributions relevant for accurate cell localization.
This directly addresses our main research question and suggests that synthetic data can be a valuable asset in training robust cell detection models, particularly when labeled real data is limited.
The slight performance enhancement observed at mAP\@50 for some models might be attributed to the increased variability introduced by the synthetic data generation process, potentially improving model generalization and robustness to variations in real-world microscopy images.

However, the subtle performance decrease observed at higher IoU thresholds (mAP\@75, mAP\@50:95) for models trained with higher proportions of synthetic data indicates a potential limitation in the fidelity of synthetic images for fine cell boundary details and precise localization.
This suggests that while diffusion models excel at generating perceptually realistic images capturing overall cell appearance and context, replicating the subtle nuances of cell morphology and edge definition present in real microscopy images remains a challenge for future refinement.
The observed sensitivity of RT-DETR models to synthetic data proportion also warrants further investigation, potentially indicating architectural differences in how transformer-based models learn from and generalize with synthetic data compared to CNN-based YOLO models.

The implications of our findings for biological research and applications are two-fold.
The demonstrated ability to generate high-quality synthetic brightfield microscopy images and effectively utilize them for training cell detection  models opens up new avenues for addressing data scarcity and annotation bottlenecks in microscopy image analysis.
Synthetic data generation offers a cost-effective and scalable approach to augment limited real datasets, potentially democratizing access to advanced cell detection techniques, particularly for research groups with limited biological resources or access to large annotated datasets.
Furthermore, the use of synthetic data can reduce the dependence on time-consuming and expensive manual labor work, saving time and resources researcher usually have to invest in the laboratorial process.
The potential to reduce or eliminate the need for fluorescent labels in certain applications, by relying on robust detection models trained with label-free brightfield images augmented by synthetic data, also aligns with ethical considerations and best practices in cell biology, minimizing potential cytotoxic effects and enabling more physiologically relevant live-cell imaging studies.
Moreover, synthetic data generation offers the unique capability to create diverse datasets encompassing rare cell phenotypes or challenging imaging conditions that are difficult to obtain in real-world experiments, potentially enhancing the robustness nad generalizability of cell detection models for a wider range of biological applications and scenarios.
Transparency in reporting the use of synthetic data in research is crucial for maintaining scientific rigor and ethical standards, ensuring appropriate interpretation and validation of findings derived from models trained with synthetic data augmentation.