\section{Discussion}
\label{sec:discussion}
The expert survey results provide compelling evidence for the high perceptual realism of brightfield microscopy images generated by our unconditional diffusion model.
The near-chance level accuracy achieved by microscopy experts in distinguishing synthetic images from real ones strongly support the notion that diffusion models can effectively synthesize microscopy data that is visually indistinguishable from real-world acquisitions.
This finding directly addresses our first sub-research question and highlights the potential of diffusion models to generate data suitable for augmenting or even substituting real microscopy images in certain applications.

Object detection experiments further reveal the practical utility of diffusion-based synthetic data for single cell detection, and importantly, also highlight the benefits of increasing dataset size through synthetic data augmentation.
The comparable, and in some cases, improved performance of models trained with synthetic data augmentation, both when using \textbf{replacement datasets} (where synthetic data replaced parts of the real data at 10\%, 30\%, and 50\% ratios) and when using \textbf{augmentation datasets} (where synthetic data was added to real data at 10\%, 30\%, and 50\% ratios), at mAP\@50 demonstrates that synthetic images can effectively capture essential cell features and spatial distributions relevant for accurate cell localization.
This directly addresses our main research question and suggests that synthetic data can be a valuable asset in training robust cell detection models, particularly when labeled real data is limited.

Specifically, the results from the \textbf{augmentation datasets} underscore the advantage of dataset size increase through synthetic data.
Unlike the \textbf{replacement datasets} where synthetic data substituted real images, the \textbf{augmentation datasets} maintained the original real dataset size and expanded it with synthetic images.
We observed that CNN-based models trained on these \textbf{augmentation datasets} consistently achieved performance at least comparable to, and often slightly better than, the baseline real-data dataset across various metrics, especially mAP\@50 and mAP\@75.
This improvement, although marginal in some cases, is notable considering the increased dataset size without additional real image acquisition effort.
The larger datasets, incorporating synthetic images, likely provided a more diverse training set, allowing the models to learn more robust features and generalize slightly better to the unseen test data.
This is particularly evident in the consistently maintained or slightly improved mAP\@50 and mAP\@75 values across different model architectures when trained on \textbf{augmentation datasets}, indicating that the added synthetic data effectively contributed to model learning without introducing detrimental artifacts or biases that would negatively impact detection accuracy at lower IoU thresholds.

The subtle performance enhancement observed at mAP\@50 for some models, and more consistently at mAP\@75 for \textbf{augmentation datasets} with 30\% synthetic data in several instances, might be attributed to a combination of factors.
The increased variability introduced by the synthetic data generation process could contribute to improved model generalization and robustness to variations in real-world microscopy images.
Furthermore, the larger dataset size in \textbf{augmentation dataset} experiments inherently provides more training examples, potentially leading to better parameter optimization and a more refined decision boundary for cell detection.
This is consistent with the general principle in deep learning that larger datasets often lead to improved model performance, especially when the added data is of sufficient quality and relevance, as demonstrated by the expert survey confirming the realism of our synthetic images.

However, similar to the \textbf{replacement dataset} experiments, the subtle performance decrease observed at higher IoU thresholds (mAP\@75, mAP\@50:95) for models trained with higher proportions of synthetic data, even in the \textbf{augmentation datasets}, indicates a potential limitation in the fidelity of synthetic images for fine cell boundary details and precise localization.
While adding synthetic data generally benefits performance, especially at lower IoUs and by increasing dataset size, the inherent limitations in replicating all nuances of real microscopy data remain relevant for tasks requiring high localization precision.
The observed sensitivity of RT-DETR models to synthetic data proportion, which persists even with added synthetic data in \textbf{augmentation datasets}, also warrants further investigation, potentially indicating architectural differences in how transformer-based models leverage larger datasets and generalize with synthetic data compared to CNN-based YOLO models.

The implications of our findings for biological research and applications are significant.
The demonstrated ability to generate high-quality synthetic brightfield microscopy images and effectively utilize them for training cell detection models, particularly by leveraging the benefits of increased dataset size without real data acquisition, opens up new avenues for addressing data scarcity and annotation bottlenecks in microscopy image analysis.
Synthetic data generation offers a cost-effective and scalable approach to augment limited real datasets, potentially democratizing access to advanced cell detection techniques, particularly for research groups with limited resources.
Furthermore, the use of synthetic data can reduce the dependence on time-consuming and expensive manual labor work, saving time and resources researchers usually have to invest in the laboratorial process.
Once a diffusion model is trained, generating synthetic images much less computationally expensive and time-consuming and can be done at scale, providing a valuable resource for training deep learning models in microscopy.
The potential to reduce or eliminate the need for fluorescent labels in certain applications, by relying on robust detection models trained with label-free brightfield images augmented by synthetic data, also aligns with ethical considerations and best practices in cell biology, minimizing potential cytotoxic effects and enabling more physiologically relevant live-cell imaging studies.
Moreover, synthetic data generation offers the unique capability to create diverse datasets encompassing rare cell phenotypes or challenging imaging conditions that are difficult to obtain in real-world experiments, potentially enhancing the robustness and generalizability of cell detection models for a wider range of biological applications and scenarios.
Transparency in reporting the use of synthetic data in research is crucial for maintaining scientific rigor and ethical standards, ensuring appropriate interpretation and validation of findings derived from models trained with synthetic data augmentation.