\section{Introduction}
\label{sec:introduction}
Single cell detection in microscopy images is a fundamental task in biological and medical applications, enabling quantitative analysis of cellular processes and disease mechanisms~\cite{meijering_cell_2012}.
Brightfield microscopy, a label-free technique, is widely used but presents challenges for automated analysis due to low contrast and variability in cell appearance~\cite{jyrki_selinummi_bright_2009}.
Deep learning, particularly CNN-based object detectors, has shown great promise for automatic cell detection~\cite{erick_moen_deep_2019,thorsten_falk_u-net_2019}.
However, the performance of these models is heavily dependent on large, annotated datasets, which are expensive and time-consuming to acquire in microscopy~\cite{ronneberger_u-net_2015}.

Synthetic data generation offers a potential solution to alleviate data scarcity.
Recent advancements in diffusion models have demonstrated their ability to generate high-fidelity images across various domains~\cite{ho_denoising_2020,song_denoising_2020}.
In microscopy, synthetic data could augment real datasets, improve model generalization, and reduce annotation efforts~\cite{rajaram_simucell_2012,trampert_deep_2021,lehmussola_synthetic_2008}.

This paper explores the use of unconditional diffusion models for generating synthetic brightfield microscopy images and investigates their impact on single cell detection accuracy.
We address the central research question: \textbf{Can synthetic brightfield microscopy images generated by diffusion models enhance the performance of object detection models for single cell detection?}
We further investigate: (1) the perceptual realism of generated images through expert evaluation, and (2) the influence of synthetic data proportion in training datasets.

Our contributions include: (i) application of unconditional diffusion models for realistic brightfield microscopy image synthesis;
(ii) a comprehensive evaluation of synthetic data augmentation for single cell detection using state-of-the-art object detectors (YOLOv9, YOLOv9, RT-DETR);
(iii) expert validation of generated image realism;
and (iv) insights into the potential and limitations of diffusion-based synthetic data for microscopy image analysis.
